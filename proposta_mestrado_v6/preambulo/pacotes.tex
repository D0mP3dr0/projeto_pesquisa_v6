% pacotes.tex
% Pacotes essenciais para a proposta de mestrado: 
% "Modelagem Preditiva de Fluxos Populacionais com Grafos Neurais"

% Codificação e idioma
\usepackage[utf8]{inputenc} % Codificação dos caracteres
\usepackage[T1]{fontenc}    % Codificação da fonte
\usepackage[brazil]{babel}  % Idioma português

\usepackage{lmodern}        % Fonte Latin Modern
\usepackage{microtype}      % Melhora a tipografia

% Pacotes matemáticos e de símbolos
\usepackage{amsmath, amssymb, amsthm}    % Fórmulas matemáticas e teoremas
\usepackage{mathtools}

% Layout e formatação geral
\usepackage{geometry}       % Configuração de margens e tamanho da página
\usepackage{setspace}       % Controle de espaçamento entre linhas
\usepackage{indentfirst}    % Indenta o primeiro parágrafo de cada seção
\usepackage{fancyhdr}       % Cabeçalhos e rodapés customizados

% Figuras e gráficos
\usepackage{graphicx}       % Inclusão de imagens
\usepackage{float}          % Controle de figuras e tabelas
\usepackage{caption}        % Personalização de legendas
\usepackage{subcaption}     % Subfiguras e subtabelas
\usepackage{wrapfig}        % Imagens com texto envolvente

% Tabelas
\usepackage{array}          % Ferramentas avançadas para tabelas
\usepackage{booktabs}       % Linhas de tabelas mais elegantes
\usepackage{multirow}       % Mesclagem de células em tabelas

% Referências bibliográficas e citações
\usepackage{natbib}         % Estilo de citações
\usepackage{doi}            % Para exibição de DOIs

% Glossário e acrônimos - deve estar antes do hyperref
\usepackage[acronym,nomain,nonumberlist,nogroupskip,toc]{glossaries}
\makeglossaries

% Hyperlinks e referências internas
\PassOptionsToPackage{hidelinks}{hyperref} % Garantir opções consistentes
\usepackage{hyperref}  % Links clicáveis sem caixas colorida

% Algoritmos e pseudocódigo
\usepackage{algorithm}      
\usepackage{algorithmic}

% Listas personalizadas
\usepackage{enumitem}

% Código fonte e destaque de sintaxe
\usepackage{listings}
\usepackage{xcolor}

% Configuração para listagens (exemplo com Python)
\lstset{
    basicstyle=\ttfamily\small,
    breaklines=true,
    frame=single,
    language=Python,
    numbers=left,
    numberstyle=\tiny,
    keywordstyle=\color{blue},
    stringstyle=\color{red},
    commentstyle=\color{gray},
    inputencoding=utf8/latin1,
    extendedchars=true,
    literate={á}{{\'a}}1 {é}{{\'e}}1 {í}{{\'i}}1 {ó}{{\'o}}1 {ú}{{\'u}}1
            {à}{{\`a}}1 {è}{{\`e}}1 {ì}{{\`i}}1 {ò}{{\`o}}1 {ù}{{\`u}}1
            {ã}{{\~a}}1 {ẽ}{{\~e}}1 {ĩ}{{\~i}}1 {õ}{{\~o}}1 {ũ}{{\~u}}1
            {ç}{{\c{c}}}1 {Ç}{{\c{C}}}1 {ê}{{\^e}}1 {ô}{{\^o}}1
            {º}{{\textordmasculine}}1 {ª}{{\textordfeminine}}1
}

% Outras utilidades
\usepackage{csquotes}  % Citações corretamente formatadas

% Corrigir problemas com nomes das tabelas em português
\addto\captionsbrazil{%
  \renewcommand{\listfigurename}{Lista de Figuras}%
  \renewcommand{\listtablename}{Lista de Tabelas}%
}
