% Resumo em português
\begin{resumo}
Esta pesquisa propõe o desenvolvimento de um modelo preditivo integrado, baseado em Redes Neurais de Grafos (GNN), para identificar e simular padrões de mobilidade populacional em ambientes urbanos. O framework integra dados de estações de rádio base (ERB), informações de terreno, dados climáticos e variáveis socioeconômicas para construir um grafo multicamadas que captura interações espaciais e temporais essenciais para a previsão de comportamentos em situações de emergência. A metodologia emprega arquiteturas avançadas de GNN como Graph Convolutional Networks (GCN), Graph Attention Networks (GAT) e Temporal Graph Networks (TGN), permitindo a extração de padrões complexos das múltiplas dimensões da mobilidade urbana. Os resultados esperados incluem: uma ferramenta robusta para previsão de fluxos populacionais, subsídios quantitativos para alocação eficiente de recursos e otimização de rotas de evacuação, e uma metodologia integrativa que combina geoprocessamento e inteligência artificial. O framework visa contribuir para o aprimoramento da capacidade preditiva dos sistemas de gestão de crises urbanas, especialmente em cenários relacionados a ameaças Químicas, Biológicas, Radiológicas e Nucleares (QBRN), oferecendo subsídios para políticas públicas e estratégias operacionais em contextos de emergência.

\vspace{\onelineskip}

\noindent
\textbf{Palavras-chave}: Grafos Neurais. Fluxos Populacionais. Geoprocessamento. Emergências Urbanas. QBRN.
\end{resumo}
