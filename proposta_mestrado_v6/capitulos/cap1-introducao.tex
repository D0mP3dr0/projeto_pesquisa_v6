\chapter{Introdução}
\label{chap:introducao}

A crescente complexidade das dinâmicas urbanas e a maior vulnerabilidade a eventos críticos exigem novas abordagens analíticas capazes de processar grandes volumes de dados geoespaciais e gerar previsões precisas. Este trabalho propõe um framework inovador baseado em Redes Neurais de Grafos (\gls{GNN}) para modelar e simular fluxos populacionais em ambientes urbanos. Integrando dados de estações de rádio base (\gls{ERB}), informações de terreno, dados climáticos e variáveis socioeconômicas, o sistema constrói um grafo multicamadas que captura interações espaciais e temporais essenciais para a previsão de comportamentos em situações de emergência.

\section{Contextualização do Problema}
As cidades contemporâneas enfrentam desafios significativos na gestão da mobilidade urbana, na resposta a desastres naturais e na organização de evacuações durante crises. Os dados provenientes de \gls{ERB}s, quando combinados com informações geográficas detalhadas, permitem a construção de grafos que representam a estrutura urbana de forma dinâmica. Esta representação facilita a identificação de áreas de alta concentração populacional e pontos críticos para intervenção, aspectos fundamentais para o planejamento estratégico e o gerenciamento eficiente de crises.

\section{Justificativa}
O modelo proposto supera as limitações dos métodos tradicionais de análise espacial, que frequentemente se restringem à análise estatística convencional, ao explorar o poder dos algoritmos de \gls{GNN}. Com arquiteturas avançadas como Graph Convolutional Networks (GCN), Graph Attention Networks (GAT) e Temporal Graph Networks (TGN), o framework permite extrair padrões complexos e integrar múltiplas dimensões (espacial, temporal e funcional) da mobilidade urbana. Esta abordagem é particularmente relevante para o desenvolvimento de estratégias de evacuação e resposta a emergências, especialmente em cenários \gls{QBRN}.

\section{Objetivos}
\subsection{Objetivo Geral}
Desenvolver um modelo preditivo integrado, baseado em Redes Neurais de Grafos, capaz de identificar e simular padrões de mobilidade populacional em ambientes urbanos, contribuindo para o aprimoramento do planejamento e da gestão de crises.

\subsection{Objetivos Específicos}
\begin{itemize}[noitemsep]
    \item Construir um grafo multicamadas representativo do ambiente urbano utilizando dados de \gls{ERB}s, informações de terreno, clima e variáveis socioeconômicas.
    \item Implementar e treinar modelos preditivos baseados em \gls{GNN}s para capturar a dinâmica dos fluxos populacionais.
    \item Desenvolver técnicas de pré-processamento e engenharia de atributos para a criação de features representativas.
    \item Validar o modelo através de simulações que identifiquem áreas críticas e otimizem rotas de evacuação em cenários emergenciais.
\end{itemize}

\section{Contribuições Esperadas}
O framework proposto deverá proporcionar:
\begin{itemize}[noitemsep]
    \item Uma ferramenta robusta para previsão de fluxos populacionais, aprimorando a resposta a emergências urbanas.
    \item Subsídios quantitativos para a alocação eficiente de recursos e otimização de rotas de evacuação.
    \item Uma metodologia integrativa que combina geoprocessamento e inteligência artificial, servindo como base para futuras pesquisas na área.
\end{itemize}

\section{Organização do Trabalho}
Este documento está estruturado da seguinte forma:
\begin{itemize}[noitemsep]
    \item No Capítulo \ref{chap:introducao}, são apresentados a contextualização, justificativa e os objetivos da pesquisa.
    \item O Capítulo \ref{chap:revisao} aborda a revisão bibliográfica e os fundamentos teóricos.
    \item O Capítulo \ref{chap:objetivos} descreve a metodologia adotada e os objetivos específicos para a construção do modelo preditivo.
    \item O Capítulo \ref{chap:cronograma} apresenta o cronograma de atividades e os recursos necessários para a execução do projeto.
\end{itemize}

\bigskip

\noindent \textbf{Palavras-chave}: Grafos Neurais, Fluxos Populacionais, Geoprocessamento, Emergências Urbanas, \gls{QBRN}.
