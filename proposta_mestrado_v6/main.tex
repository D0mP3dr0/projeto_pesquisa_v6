% !TEX encoding = UTF-8
% ARQUIVO MAIN.TEX - PROPOSTA DE MESTRADO
% Modelagem Preditiva de Fluxos Populacionais com Grafos Neurais
% -------------------------------------------------------------

\documentclass[
  12pt,               % Tamanho da fonte
  a4paper,            % Tamanho do papel
  oneside,            % Impressão em apenas um lado do papel
  openright,          % Capítulos começam em página ímpar
  brazil,             % Idioma principal
  chapter=TITLE,      % Formatação de capítulos em caixa alta
]{abntex2}

% Garantir compatibilidade com caracteres UTF-8
\DeclareUnicodeCharacter{00A0}{ }
\DeclareUnicodeCharacter{00AD}{-}

% -----------------------------------------------------------------
% PREÂMBULO - IMPORTAÇÃO DE PACOTES E CONFIGURAÇÕES
% -----------------------------------------------------------------
% pacotes.tex
% Pacotes essenciais para a proposta de mestrado: 
% "Modelagem Preditiva de Fluxos Populacionais com Grafos Neurais"

% Codificação e idioma
\usepackage[utf8]{inputenc} % Codificação dos caracteres
\usepackage[T1]{fontenc}    % Codificação da fonte
\usepackage[brazil]{babel}  % Idioma português

\usepackage{lmodern}        % Fonte Latin Modern
\usepackage{microtype}      % Melhora a tipografia

% Pacotes matemáticos e de símbolos
\usepackage{amsmath, amssymb, amsthm}    % Fórmulas matemáticas e teoremas
\usepackage{mathtools}

% Layout e formatação geral
\usepackage{geometry}       % Configuração de margens e tamanho da página
\usepackage{setspace}       % Controle de espaçamento entre linhas
\usepackage{indentfirst}    % Indenta o primeiro parágrafo de cada seção
\usepackage{fancyhdr}       % Cabeçalhos e rodapés customizados

% Figuras e gráficos
\usepackage{graphicx}       % Inclusão de imagens
\usepackage{float}          % Controle de figuras e tabelas
\usepackage{caption}        % Personalização de legendas
\usepackage{subcaption}     % Subfiguras e subtabelas
\usepackage{wrapfig}        % Imagens com texto envolvente

% Tabelas
\usepackage{array}          % Ferramentas avançadas para tabelas
\usepackage{booktabs}       % Linhas de tabelas mais elegantes
\usepackage{multirow}       % Mesclagem de células em tabelas

% Referências bibliográficas e citações
\usepackage{natbib}         % Estilo de citações
\usepackage{doi}            % Para exibição de DOIs

% Glossário e acrônimos - deve estar antes do hyperref
\usepackage[acronym,nomain,nonumberlist,nogroupskip,toc]{glossaries}
\makeglossaries

% Hyperlinks e referências internas
\PassOptionsToPackage{hidelinks}{hyperref} % Garantir opções consistentes
\usepackage{hyperref}  % Links clicáveis sem caixas colorida

% Algoritmos e pseudocódigo
\usepackage{algorithm}      
\usepackage{algorithmic}

% Listas personalizadas
\usepackage{enumitem}

% Código fonte e destaque de sintaxe
\usepackage{listings}
\usepackage{xcolor}

% Configuração para listagens (exemplo com Python)
\lstset{
    basicstyle=\ttfamily\small,
    breaklines=true,
    frame=single,
    language=Python,
    numbers=left,
    numberstyle=\tiny,
    keywordstyle=\color{blue},
    stringstyle=\color{red},
    commentstyle=\color{gray},
    inputencoding=utf8/latin1,
    extendedchars=true,
    literate={á}{{\'a}}1 {é}{{\'e}}1 {í}{{\'i}}1 {ó}{{\'o}}1 {ú}{{\'u}}1
            {à}{{\`a}}1 {è}{{\`e}}1 {ì}{{\`i}}1 {ò}{{\`o}}1 {ù}{{\`u}}1
            {ã}{{\~a}}1 {ẽ}{{\~e}}1 {ĩ}{{\~i}}1 {õ}{{\~o}}1 {ũ}{{\~u}}1
            {ç}{{\c{c}}}1 {Ç}{{\c{C}}}1 {ê}{{\^e}}1 {ô}{{\^o}}1
            {º}{{\textordmasculine}}1 {ª}{{\textordfeminine}}1
}

% Outras utilidades
\usepackage{csquotes}  % Citações corretamente formatadas

% Corrigir problemas com nomes das tabelas em português
\addto\captionsbrazil{%
  \renewcommand{\listfigurename}{Lista de Figuras}%
  \renewcommand{\listtablename}{Lista de Tabelas}%
}
         % Importação de pacotes necessários
% Arquivo de correção de problemas de codificação
% Adiciona suporte para caracteres especiais que podem causar problemas

% Caracteres problemáticos específicos do projeto
\DeclareUnicodeCharacter{00A0}{ }     % Espaço indivisível
\DeclareUnicodeCharacter{00AD}{-}     % Hífen suave
\DeclareUnicodeCharacter{00B4}{'}     % Acento agudo
\DeclareUnicodeCharacter{00A7}{\S}    % Símbolo de seção
\DeclareUnicodeCharacter{00A9}{\copyright} % Copyright
\DeclareUnicodeCharacter{00AE}{\textregistered} % Marca registrada
\DeclareUnicodeCharacter{00BA}{\textordmasculine} % Ordinal masculino
\DeclareUnicodeCharacter{00AA}{\textordfeminine} % Ordinal feminino
\DeclareUnicodeCharacter{00B0}{\textdegree} % Grau
\DeclareUnicodeCharacter{00B1}{\textpm} % Mais ou menos
\DeclareUnicodeCharacter{00B2}{\texttwosuperior} % Ao quadrado
\DeclareUnicodeCharacter{00B3}{\textthreesuperior} % Ao cubo

% Caracteres acentuados comuns em português
\DeclareUnicodeCharacter{00E1}{\'{a}} % á
\DeclareUnicodeCharacter{00E9}{\'{e}} % é
\DeclareUnicodeCharacter{00ED}{\'{i}} % í
\DeclareUnicodeCharacter{00F3}{\'{o}} % ó
\DeclareUnicodeCharacter{00FA}{\'{u}} % ú
\DeclareUnicodeCharacter{00E0}{\`{a}} % à
\DeclareUnicodeCharacter{00E8}{\`{e}} % è
\DeclareUnicodeCharacter{00F2}{\`{o}} % ò
\DeclareUnicodeCharacter{00E2}{\^{a}} % â
\DeclareUnicodeCharacter{00EA}{\^{e}} % ê
\DeclareUnicodeCharacter{00EE}{\^{i}} % î
\DeclareUnicodeCharacter{00F4}{\^{o}} % ô
\DeclareUnicodeCharacter{00FB}{\^{u}} % û
\DeclareUnicodeCharacter{00E3}{\~{a}} % ã
\DeclareUnicodeCharacter{00F5}{\~{o}} % õ
\DeclareUnicodeCharacter{00E7}{\c{c}} % ç
\DeclareUnicodeCharacter{00C7}{\c{C}} % Ç

% Letras maiúsculas acentuadas
\DeclareUnicodeCharacter{00C1}{\'{A}} % Á
\DeclareUnicodeCharacter{00C9}{\'{E}} % É
\DeclareUnicodeCharacter{00CD}{\'{I}} % Í
\DeclareUnicodeCharacter{00D3}{\'{O}} % Ó
\DeclareUnicodeCharacter{00DA}{\'{U}} % Ú
\DeclareUnicodeCharacter{00C0}{\`{A}} % À
\DeclareUnicodeCharacter{00C8}{\`{E}} % È
\DeclareUnicodeCharacter{00D2}{\`{O}} % Ò
\DeclareUnicodeCharacter{00C2}{\^{A}} % Â
\DeclareUnicodeCharacter{00CA}{\^{E}} % Ê
\DeclareUnicodeCharacter{00CE}{\^{I}} % Î
\DeclareUnicodeCharacter{00D4}{\^{O}} % Ô
\DeclareUnicodeCharacter{00DB}{\^{U}} % Û
\DeclareUnicodeCharacter{00C3}{\~{A}} % Ã
\DeclareUnicodeCharacter{00D5}{\~{O}} % Õ}  % Correções de codificação
% Configurações gerais do documento LaTeX
% Proposta de Mestrado - Modelagem Preditiva de Fluxos Populacionais com Grafos Neurais
% Autor: Luis Felipe Comodo Seelig

% Idiomas
\selectlanguage{brazil}

% Configurações do ABNTeX2
\renewcommand{\ABNTEXchapterfont}{\rmfamily\bfseries}
\renewcommand{\ABNTEXsectionfont}{\rmfamily\bfseries}

% Configurações de espaçamento
\setlength{\parindent}{1.5cm}
\setlength{\parskip}{0.2cm}

% Configurações para cabeçalhos e rodapés
\renewcommand{\headrulewidth}{0pt}
\renewcommand{\footrulewidth}{0pt}

% Configurações específicas para citações
% \ABNTEXbibliographytype{alf} - Removido, comando não definido

% Configurações para lista de siglas (acrônimos)
% Os acrônimos são definidos no arquivo acronimos.tex

% Configuração para sumário
\setcounter{tocdepth}{4}
\setcounter{secnumdepth}{4}

% Configurações para ambiente de itemize e enumerate
\renewcommand\labelitemi{$\bullet$}
\renewcommand\labelitemii{$\circ$}

% Configurações para tabelas e figuras
\captionsetup[table]{skip=10pt, font=footnotesize}
\captionsetup[figure]{skip=10pt, font=footnotesize}

% Configurações para hyperlinks
\hypersetup{
    colorlinks=true,
    linkcolor=blue,
    citecolor=blue,
    filecolor=magenta,
    urlcolor=blue,
    pdftitle={Modelagem Preditiva de Fluxos Populacionais com Grafos Neurais},
    pdfauthor={Luis Felipe Comodo Seelig},
    pdfsubject={Proposta de Mestrado},
    pdfkeywords={GNN, Fluxos Populacionais, Grafos, Emergências}
}

% Configurações para nomenclatura
% Comentado pois requer o pacote nomencl que não está carregado
% \makenomenclature
% \renewcommand{\nomname}{Lista de Símbolos}

% Ativar numeração de equações por capítulo
\numberwithin{equation}{chapter}
   % Configurações gerais do documento
% ===================================================================
% ACRÔNIMOS - Modelagem Preditiva de Fluxos Populacionais com Grafos Neurais
% ===================================================================
% Autor: Luis Felipe Comodo Seelig
% ===================================================================

% A definição do novo glossário foi movida para o arquivo pacotes.tex
\makeglossaries

% ===================================================================
% REDES NEURAIS E GRAFOS
% ===================================================================

\newacronym{GNN}{GNN}{Graph Neural Network (Rede Neural de Grafos)}
\newacronym{GCN}{GCN}{Graph Convolutional Network (Rede Neural Convolucional de Grafos)}
\newacronym{GAT}{GAT}{Graph Attention Network (Rede Neural de Grafos com Atenção)}
\newacronym{TGN}{TGN}{Temporal Graph Network (Rede Neural de Grafos Temporal)}
\newacronym{DCRNN}{DCRNN}{Diffusion Convolutional Recurrent Neural Network (Rede Neural Recorrente Convolucional de Difusão)}
\newacronym{CNN}{CNN}{Convolutional Neural Network (Rede Neural Convolucional)}
\newacronym{RNN}{RNN}{Recurrent Neural Network (Rede Neural Recorrente)}
\newacronym{LSTM}{LSTM}{Long Short-Term Memory (Memória de Longo-Curto Prazo)}
\newacronym{GRU}{GRU}{Gated Recurrent Unit (Unidade Recorrente com Portões)}
\newacronym{XAI}{XAI}{Explainable Artificial Intelligence (Inteligência Artificial Explicável)}
\newacronym{PyG}{PyG}{PyTorch Geometric (Biblioteca para GNNs)}
\newacronym{ML}{ML}{Machine Learning (Aprendizado de Máquina)}
\newacronym{AI}{AI}{Artificial Intelligence (Inteligência Artificial)}
\newacronym{DL}{DL}{Deep Learning (Aprendizado Profundo)}
\newacronym{GDL}{GDL}{Geometric Deep Learning (Aprendizado Profundo Geométrico)}

% ===================================================================
% GEOPROCESSAMENTO E ANÁLISE ESPACIAL
% ===================================================================

\newacronym{GIS}{GIS}{Geographic Information System (Sistema de Informação Geográfica)}
\newacronym{GeoAI}{GeoAI}{Geospatial Artificial Intelligence (Inteligência Artificial Geoespacial)}
\newacronym{SIG}{SIG}{Sistema de Informação Geográfica}
\newacronym{ERB}{ERB}{Estação Rádio Base}
\newacronym{SRTM}{SRTM}{Shuttle Radar Topography Mission}
\newacronym{GDAL}{GDAL}{Geospatial Data Abstraction Library}
\newacronym{OSM}{OSM}{OpenStreetMap}
\newacronym{GPS}{GPS}{Global Positioning System (Sistema de Posicionamento Global)}
\newacronym{GNSS}{GNSS}{Global Navigation Satellite System (Sistema Global de Navegação por Satélite)}
\newacronym{POI}{POI}{Point of Interest (Ponto de Interesse)}
\newacronym{MDE}{MDE}{Modelo Digital de Elevação}
\newacronym{SIRGAS}{SIRGAS}{Sistema de Referência Geocêntrico para as Américas}

% ===================================================================
% TERMOS MILITARES E DE PLANEJAMENTO
% ===================================================================

\newacronym{PITCIC}{PITCIC}{Processo de Integração Terreno, Condições Meteorológicas, Inimigo e Considerações Civis}
\newacronym{QBRN}{QBRN}{Químico, Biológico, Radiológico e Nuclear}
\newacronym{IPB}{IPB}{Intelligence Preparation of the Battlefield (Preparação de Inteligência do Campo de Batalha)}
\newacronym{COTER}{COTER}{Comando de Operações Terrestres}
\newacronym{EB}{EB}{Exército Brasileiro}
\newacronym{IME}{IME}{Instituto Militar de Engenharia}
\newacronym{CMSE}{CMSE}{Comando Militar do Sudeste}
\newacronym{PROSUB}{PROSUB}{Programa de Submarinos da Marinha}
\newacronym{AMAN}{AMAN}{Academia Militar das Agulhas Negras}
\newacronym{EsAO}{EsAO}{Escola de Aperfeiçoamento de Oficiais}
\newacronym{SSCI}{SSCI}{Subseção de Contrainteligência}

% ===================================================================
% TERMOS DE EMERGÊNCIA E GESTÃO DE CRISE
% ===================================================================

\newacronym{CEMADEN}{CEMADEN}{Centro Nacional de Monitoramento e Alertas de Desastres Naturais}
\newacronym{INMET}{INMET}{Instituto Nacional de Meteorologia}
\newacronym{CRED}{CRED}{Centre for Research on the Epidemiology of Disasters}
\newacronym{FEMA}{FEMA}{Federal Emergency Management Agency}

% ===================================================================
% INSTITUIÇÕES E ORGANIZAÇÕES
% ===================================================================

\newacronym{UFABC}{UFABC}{Universidade Federal do ABC}
\newacronym{USP}{USP}{Universidade de São Paulo}
\newacronym{ESALQ}{ESALQ}{Escola Superior de Agricultura Luiz de Queiroz}
\newacronym{UTFPR}{UTFPR}{Universidade Tecnológica Federal do Paraná}
\newacronym{IBGE}{IBGE}{Instituto Brasileiro de Geografia e Estatística}
\newacronym{IPEA}{IPEA}{Instituto de Pesquisa Econômica Aplicada}
\newacronym{ANATEL}{ANATEL}{Agência Nacional de Telecomunicações}

% ===================================================================
% FORMATOS E PADRÕES DE DADOS
% ===================================================================

\newacronym{JSON}{JSON}{JavaScript Object Notation}
\newacronym{CSV}{CSV}{Comma-Separated Values}
\newacronym{GPKG}{GPKG}{GeoPackage}
\newacronym{API}{API}{Application Programming Interface}
\newacronym{HTTP}{HTTP}{Hypertext Transfer Protocol}
\newacronym{SQL}{SQL}{Structured Query Language}
\newacronym{NoSQL}{NoSQL}{Not Only SQL}
\newacronym{WKT}{WKT}{Well-Known Text}       % Definição de acrônimos utilizados no texto

% -----------------------------------------------------------------
% INFORMAÇÕES DO DOCUMENTO (Título, autor, orientador, etc.)
% -----------------------------------------------------------------
\titulo{Modelagem Preditiva de Fluxos Populacionais com Grafos Neurais: Análise Retroativa e Simulação de Movimentos em Cenários de Crises}
\autor{LUIS FELIPE COMODO SEELIG}
\orientador{Professor Doutor} % Adicionar nome do orientador
\instituicao{Instituto Militar de Engenharia}
\local{Rio de Janeiro}
\data{\the\year}
\tipotrabalho{Proposta de Dissertação}

% -----------------------------------------------------------------
% INÍCIO DO DOCUMENTO
% -----------------------------------------------------------------
\begin{document}

% -----------------------------------------------------------------
% ELEMENTOS PRÉ-TEXTUAIS (Simplificados para uma proposta)
% -----------------------------------------------------------------
% Capa segundo as normas do ABNTeX2
\imprimircapa         % Capa do trabalho

\pretextual
\imprimirfolhaderosto            % Folha de rosto (usando comando interno do ABNTeX2)
% Resumo em português
\begin{resumo}
Esta pesquisa propõe o desenvolvimento de um modelo preditivo integrado, baseado em Redes Neurais de Grafos (GNN), para identificar e simular padrões de mobilidade populacional em ambientes urbanos. O framework integra dados de estações de rádio base (ERB), informações de terreno, dados climáticos e variáveis socioeconômicas para construir um grafo multicamadas que captura interações espaciais e temporais essenciais para a previsão de comportamentos em situações de emergência. A metodologia emprega arquiteturas avançadas de GNN como Graph Convolutional Networks (GCN), Graph Attention Networks (GAT) e Temporal Graph Networks (TGN), permitindo a extração de padrões complexos das múltiplas dimensões da mobilidade urbana. Os resultados esperados incluem: uma ferramenta robusta para previsão de fluxos populacionais, subsídios quantitativos para alocação eficiente de recursos e otimização de rotas de evacuação, e uma metodologia integrativa que combina geoprocessamento e inteligência artificial. O framework visa contribuir para o aprimoramento da capacidade preditiva dos sistemas de gestão de crises urbanas, especialmente em cenários relacionados a ameaças Químicas, Biológicas, Radiológicas e Nucleares (QBRN), oferecendo subsídios para políticas públicas e estratégias operacionais em contextos de emergência.

\vspace{\onelineskip}

\noindent
\textbf{Palavras-chave}: Grafos Neurais. Fluxos Populacionais. Geoprocessamento. Emergências Urbanas. QBRN.
\end{resumo}
      % Resumo em português
\tableofcontents                 % Sumário

% -----------------------------------------------------------------
% ELEMENTOS TEXTUAIS
% -----------------------------------------------------------------
\textual

% Introdução
\chapter{Introdução}
\label{chap:introducao}

A crescente complexidade das dinâmicas urbanas e a maior vulnerabilidade a eventos críticos exigem novas abordagens analíticas capazes de processar grandes volumes de dados geoespaciais e gerar previsões precisas. Este trabalho propõe um framework inovador baseado em Redes Neurais de Grafos (\gls{GNN}) para modelar e simular fluxos populacionais em ambientes urbanos. Integrando dados de estações de rádio base (\gls{ERB}), informações de terreno, dados climáticos e variáveis socioeconômicas, o sistema constrói um grafo multicamadas que captura interações espaciais e temporais essenciais para a previsão de comportamentos em situações de emergência.

\section{Contextualização do Problema}
As cidades contemporâneas enfrentam desafios significativos na gestão da mobilidade urbana, na resposta a desastres naturais e na organização de evacuações durante crises. Os dados provenientes de \gls{ERB}s, quando combinados com informações geográficas detalhadas, permitem a construção de grafos que representam a estrutura urbana de forma dinâmica. Esta representação facilita a identificação de áreas de alta concentração populacional e pontos críticos para intervenção, aspectos fundamentais para o planejamento estratégico e o gerenciamento eficiente de crises.

\section{Justificativa}
O modelo proposto supera as limitações dos métodos tradicionais de análise espacial, que frequentemente se restringem à análise estatística convencional, ao explorar o poder dos algoritmos de \gls{GNN}. Com arquiteturas avançadas como Graph Convolutional Networks (GCN), Graph Attention Networks (GAT) e Temporal Graph Networks (TGN), o framework permite extrair padrões complexos e integrar múltiplas dimensões (espacial, temporal e funcional) da mobilidade urbana. Esta abordagem é particularmente relevante para o desenvolvimento de estratégias de evacuação e resposta a emergências, especialmente em cenários \gls{QBRN}.

\section{Objetivos}
\subsection{Objetivo Geral}
Desenvolver um modelo preditivo integrado, baseado em Redes Neurais de Grafos, capaz de identificar e simular padrões de mobilidade populacional em ambientes urbanos, contribuindo para o aprimoramento do planejamento e da gestão de crises.

\subsection{Objetivos Específicos}
\begin{itemize}[noitemsep]
    \item Construir um grafo multicamadas representativo do ambiente urbano utilizando dados de \gls{ERB}s, informações de terreno, clima e variáveis socioeconômicas.
    \item Implementar e treinar modelos preditivos baseados em \gls{GNN}s para capturar a dinâmica dos fluxos populacionais.
    \item Desenvolver técnicas de pré-processamento e engenharia de atributos para a criação de features representativas.
    \item Validar o modelo através de simulações que identifiquem áreas críticas e otimizem rotas de evacuação em cenários emergenciais.
\end{itemize}

\section{Contribuições Esperadas}
O framework proposto deverá proporcionar:
\begin{itemize}[noitemsep]
    \item Uma ferramenta robusta para previsão de fluxos populacionais, aprimorando a resposta a emergências urbanas.
    \item Subsídios quantitativos para a alocação eficiente de recursos e otimização de rotas de evacuação.
    \item Uma metodologia integrativa que combina geoprocessamento e inteligência artificial, servindo como base para futuras pesquisas na área.
\end{itemize}

\section{Organização do Trabalho}
Este documento está estruturado da seguinte forma:
\begin{itemize}[noitemsep]
    \item No Capítulo \ref{chap:introducao}, são apresentados a contextualização, justificativa e os objetivos da pesquisa.
    \item O Capítulo \ref{chap:revisao} aborda a revisão bibliográfica e os fundamentos teóricos.
    \item O Capítulo \ref{chap:objetivos} descreve a metodologia adotada e os objetivos específicos para a construção do modelo preditivo.
    \item O Capítulo \ref{chap:cronograma} apresenta o cronograma de atividades e os recursos necessários para a execução do projeto.
\end{itemize}

\bigskip

\noindent \textbf{Palavras-chave}: Grafos Neurais, Fluxos Populacionais, Geoprocessamento, Emergências Urbanas, \gls{QBRN}.


% Revisão da Literatura
\chapter{Revisão da Literatura}
\label{chap:revisao}

\section{Inteligência Artificial Geoespacial}
A Inteligência Artificial Geoespacial (\gls{GeoAI}) integra métodos de inteligência artificial com dados geográficos, permitindo a análise de grandes volumes de dados oriundos de sensores urbanos, \gls{ERB}s, imagens de satélite e fontes complementares. Essa abordagem possibilita extrair insights para planejamento urbano e gestão de emergências, contribuindo para o aprimoramento das estratégias de defesa e resposta a crises. Diversos estudos, como os de \cite{GAO2021}, demonstram que a aplicação de \gls{GeoAI} pode transformar registros tradicionais em informações acionáveis para a tomada de decisão.

Para o desenvolvimento deste trabalho, múltiplas fontes de dados geoespaciais serão integradas na construção do grafo multicamadas, conforme apresentado na Tabela \ref{tab:dados-geoespaciais}. A diversidade dessas fontes representa um desafio metodológico, mas também uma oportunidade para capturar a complexidade do ambiente urbano em suas múltiplas dimensões.

\section{Redes Neurais de Grafos (GNN)}
As Redes Neurais de Grafos (\gls{GNN}s) são especialmente adequadas para lidar com dados estruturados em forma de grafos, onde entidades (nós) e suas relações (arestas) representam o ambiente urbano. No contexto deste trabalho, utilizam-se principalmente três arquiteturas:
\begin{itemize}[noitemsep]
    \item \textbf{Graph Convolutional Networks (GCN)}: Capturam a estrutura espacial e permitem a extração de características locais e globais.
    \item \textbf{Graph Attention Networks (GAT)}: Introduzem mecanismos de atenção para atribuir importância diferenciada às conexões entre nós.
    \item \textbf{Temporal Graph Networks (TGN)}: Incorporam a dimensão temporal na modelagem, permitindo prever a evolução dos padrões de mobilidade.
\end{itemize}
Essas técnicas são fundamentais para a modelagem preditiva dos fluxos populacionais, uma vez que permitem a integração de informações de diferentes fontes (como terreno, clima e dados socioeconômicos) em um grafo multicamadas.

A Tabela \ref{tab:comparacao-gnn} apresenta uma comparação detalhada entre as principais arquiteturas de \gls{GNN} que serão exploradas neste trabalho. Observa-se que, enquanto GCN oferece boa escalabilidade para grafos extensos, as arquiteturas GAT e TGN proporcionam recursos avançados como mecanismos de atenção e melhor tratamento da dimensão temporal, essenciais para a modelagem de fluxos dinâmicos.

\section{Aplicações em Mobilidade Urbana e Gestão de Crises}
A previsão de fluxos populacionais por meio de \gls{GNN}s tem se destacado como uma ferramenta valiosa para o planejamento de emergências. Ao modelar a dinâmica urbana como um grafo, é possível identificar áreas de alta vulnerabilidade e pontos críticos para intervenções imediatas. Essa abordagem possibilita:
\begin{itemize}[noitemsep]
    \item Análise preditiva dos deslocamentos populacionais em tempo real;
    \item Identificação de rotas de evacuação eficientes;
    \item Suporte à tomada de decisões em cenários críticos, como eventos \gls{QBRN}.
\end{itemize}
Estudos recentes indicam que a combinação de \gls{GeoAI} e \gls{GNN} permite uma compreensão mais aprofundada das interações urbanas, oferecendo subsídios quantitativos para a gestão de emergências e o dimensionamento de recursos.

Para avaliar a eficácia dos modelos desenvolvidos, serão utilizadas métricas específicas que contemplam tanto a precisão numérica quanto a capacidade de identificação de áreas críticas, conforme detalhado na Tabela \ref{tab:metricas-desempenho}. A combinação destas métricas fornecerá uma avaliação abrangente do desempenho do sistema em diferentes cenários de aplicação.

\section{Considerações Finais}
A revisão da literatura evidencia que a integração de métodos de \gls{GeoAI} com redes neurais de grafos representa um avanço promissor para a análise de mobilidade urbana. A capacidade de construir modelos preditivos que incorporam tanto a estrutura espacial quanto a dinâmica temporal do ambiente urbano fundamenta o desenvolvimento deste trabalho, que busca oferecer um framework robusto para apoiar o planejamento e a gestão de crises em cidades modernas.

A abordagem proposta distingue-se pela combinação de diferentes técnicas e fontes de dados, resultando em um sistema integrado capaz de capturar a complexidade dos fluxos populacionais e prever comportamentos emergentes em situações críticas. Esta visão holística, sustentada por algoritmos avançados de aprendizado profundo em grafos, representa uma contribuição significativa para o campo de pesquisa em mobilidade urbana e gestão de emergências.


% Objetivos e Metodologia
\chapter{Objetivos e Metodologia}
\label{chap:objetivos}

\section{Objetivo Geral}
Desenvolver um modelo preditivo integrado, baseado em Redes Neurais de Grafos (\gls{GNN}), para identificar e simular padrões de mobilidade populacional em ambientes urbanos. O sistema visa subsidiar o planejamento de emergências e a gestão de crises, especialmente em cenários críticos como os relacionados a ameaças \gls{QBRN}.

\section{Objetivos Específicos}
\begin{itemize}[noitemsep]
    \item Construir um grafo urbano multicamadas a partir da integração de dados geoespaciais oriundos de estações de rádio base (\gls{ERB}), dados de terreno, informações climáticas e variáveis socioeconômicas.
    \item Desenvolver e treinar modelos preditivos utilizando arquiteturas de \gls{GNN}, como Graph Convolutional Networks (GCN), Graph Attention Networks (GAT) e Temporal Graph Networks (TGN), para capturar as interações espaciais e temporais da mobilidade urbana.
    \item Implementar técnicas de pré-processamento e engenharia de atributos, convertendo dados brutos em features representativas para a modelagem dos grafos.
    \item Validar o modelo por meio de simulações de cenários emergenciais e análise quantitativa (utilizando métricas como as descritas na Tabela \ref{tab:metricas-desempenho}) para identificação de áreas críticas e otimização de rotas de evacuação.
    \item Avaliar a aplicabilidade do framework como ferramenta de suporte à decisão para o planejamento urbano e a gestão de crises em contextos com riscos \gls{QBRN}.
\end{itemize}

\section{Metodologia}
A abordagem metodológica deste trabalho contempla as seguintes etapas:
\begin{enumerate}[noitemsep]
    \item \textbf{Aquisição e Pré-processamento:} Coleta de dados de \gls{ERB}, terreno, clima e variáveis socioeconômicas (conforme detalhado na Tabela \ref{tab:dados-geoespaciais}), seguida de limpeza, normalização e integração dos dados.
    
    \item \textbf{Construção do Grafo Urbano:} Modelagem do ambiente urbano como um grafo multicamadas, onde os nós representam elementos espaciais (interseções, edificações, áreas de interesse) e as arestas simbolizam conexões físicas e funcionais.
    
    \item \textbf{Desenvolvimento dos Modelos Preditivos:} Implementação e treinamento de modelos baseados em \gls{GNN} (conforme comparativo apresentado na Tabela \ref{tab:comparacao-gnn}) utilizando PyTorch Geometric, com foco em arquiteturas GCN, GAT e TGN.
    
    \item \textbf{Simulação e Validação:} Realização de simulações para prever fluxos populacionais, identificação de áreas críticas e análise do desempenho preditivo por meio de métricas quantitativas.
    
    \item \textbf{Análise e Aplicação:} Comparação dos resultados com métodos tradicionais e avaliação do potencial do modelo para subsidiar decisões estratégicas em situações de emergência.
\end{enumerate}

\section{Abordagem Experimental}
\label{sec:abordagem-experimental}

O desenvolvimento experimental do framework seguirá uma abordagem progressiva, baseada nas seguintes fases:

\begin{enumerate}[label=\alph*), noitemsep]
    \item \textbf{Fase Inicial - Prova de Conceito:} Implementação de um protótipo utilizando dados sintéticos e grafos simplificados para validar os algoritmos básicos de \gls{GNN}.
    
    \item \textbf{Fase Intermediária - Integração de Dados Reais:} Incorporação de dados reais de \gls{ERB}s e geoespaciais para construção do grafo urbano completo, com treinamento e ajuste dos modelos para capturar padrões de mobilidade.
    
    \item \textbf{Fase Avançada - Cenários Emergenciais:} Simulação de situações de emergência para testar a capacidade preditiva do modelo em condições críticas, avaliando a precisão das previsões e a eficácia das rotas de evacuação sugeridas.
    
    \item \textbf{Fase Final - Validação e Otimização:} Refinamento do modelo com base nos resultados das fases anteriores, otimização de hiperparâmetros e avaliação final do desempenho do sistema.
\end{enumerate}

Para cada fase, serão realizados experimentos controlados que permitam isolar e analisar o impacto de diferentes variáveis no desempenho do modelo, garantindo a robustez e a confiabilidade dos resultados obtidos.

\bigskip

\noindent \textbf{Contribuição:}  
O desenvolvimento deste framework integrará conceitos avançados de \gls{GeoAI} e aprendizado profundo em grafos, contribuindo para a melhoria da capacidade preditiva dos sistemas de gestão de crises urbanas e oferecendo subsídios para políticas públicas e estratégias operacionais em contextos de emergência.


% Cronograma e Recursos
\chapter{Cronograma e Recursos}
\label{chap:cronograma}

Este capítulo apresenta o planejamento temporal e os recursos necessários para a execução bem-sucedida do projeto de pesquisa. O cronograma foi estruturado considerando as etapas metodológicas descritas anteriormente e os recursos foram dimensionados para atender às demandas técnicas e computacionais do desenvolvimento do framework proposto.

\section{Cronograma de Atividades}

A Tabela \ref{tab:cronograma-atividades} apresenta a distribuição temporal das principais atividades do projeto, organizadas em etapas sequenciais que totalizam 15 meses de duração. O cronograma prevê sobreposições estratégicas entre algumas atividades, de modo a otimizar o fluxo de trabalho e garantir o cumprimento dos prazos estabelecidos.

\begin{table}[H]
    \centering
    \caption{Cronograma de Atividades do Projeto}
    \label{tab:cronograma-atividades}
    \begin{tabular}{lc}
        \toprule
        \textbf{Etapa} & \textbf{Duração (meses)} \\
        \midrule
        Revisão bibliográfica e fundamentação teórica & 2 \\
        Estruturação e refinamento do projeto de pesquisa & 1 \\
        Coleta e processamento de dados geoespaciais e \gls{ERB}s & 2 \\
        Desenvolvimento do grafo com integração do \gls{PITCIC} e da camada \gls{ERB} & 2 \\
        Implementação e treinamento dos modelos \gls{GNN}s & 2 \\
        Simulação de cenários e validação dos resultados preliminares & 1 \\
        Sistema de alerta e módulo de simulação de notificações & 1 \\
        Avaliação dos resultados e otimização do modelo & 1 \\
        Redação e revisão da Dissertação de Mestrado & 3 \\
        \bottomrule
    \end{tabular}
\end{table}

\section{Detalhamento das Etapas}

A seguir, são detalhadas as atividades previstas em cada etapa do cronograma, com seus respectivos objetivos e entregas:

\subsection{Revisão Bibliográfica e Fundamentação Teórica}
\textbf{Duração:} 2 meses

\textbf{Atividades:}
\begin{itemize}[noitemsep]
    \item Levantamento e análise de publicações científicas nas áreas de \gls{GNN}s, modelagem de fluxos populacionais e sistemas de resposta a emergências.
    \item Estudo dos fundamentos teóricos de grafos multicamadas e suas aplicações em análise geoespacial.
    \item Investigação de métodos existentes para integração de dados de \gls{ERB}s em modelagem urbana.
    \item Análise de técnicas utilizadas para previsão de comportamentos populacionais em situações de crise.
\end{itemize}

\textbf{Entregas:} Relatório de revisão bibliográfica; Mapa conceitual das principais teorias e abordagens; Banco de referências organizado.

\subsection{Estruturação e Refinamento do Projeto}
\textbf{Duração:} 1 mês

\textbf{Atividades:}
\begin{itemize}[noitemsep]
    \item Definição detalhada da metodologia e fluxo de trabalho.
    \item Seleção e justificativa das métricas de avaliação.
    \item Preparação do ambiente de desenvolvimento com as bibliotecas e frameworks necessários.
    \item Elaboração de protocolos para aquisição e tratamento de dados.
\end{itemize}

\textbf{Entregas:} Projeto detalhado; Ambiente de desenvolvimento configurado; Protocolos metodológicos documentados.

\subsection{Coleta e Processamento de Dados}
\textbf{Duração:} 2 meses

\textbf{Atividades:}
\begin{itemize}[noitemsep]
    \item Aquisição de dados brutos de \gls{ERB}s através de parcerias institucionais.
    \item Coleta de informações geoespaciais e variáveis contextuais de fontes públicas.
    \item Limpeza, normalização e transformação dos dados para formatos compatíveis.
    \item Criação de pipelines de pré-processamento automatizados.
\end{itemize}

\textbf{Entregas:} Conjunto de dados processados; Documentação dos procedimentos de tratamento; Estatísticas descritivas dos dados coletados.

\subsection{Desenvolvimento do Grafo Multicamadas}
\textbf{Duração:} 2 meses

\textbf{Atividades:}
\begin{itemize}[noitemsep]
    \item Construção da estrutura de grafos integrando o modelo \gls{PITCIC} com dados de cobertura de \gls{ERB}s.
    \item Definição e implementação da representação dos nós, arestas e atributos.
    \item Desenvolvimento de algoritmos para integração das diferentes camadas do grafo.
    \item Validação da consistência e coerência da estrutura do grafo.
\end{itemize}

\textbf{Entregas:} Biblioteca de construção de grafos; Grafo multicamadas implementado; Documentação técnica da estrutura.

\subsection{Implementação e Treinamento dos Modelos}
\textbf{Duração:} 2 meses

\textbf{Atividades:}
\begin{itemize}[noitemsep]
    \item Implementação das arquiteturas de \gls{GNN}s selecionadas (GCN, GAT, TGN) utilizando PyTorch Geometric.
    \item Configuração dos procedimentos de treinamento e validação.
    \item Execução de experimentos com diferentes configurações de hiperparâmetros.
    \item Análise comparativa do desempenho dos diferentes modelos.
\end{itemize}

\textbf{Entregas:} Modelos implementados e treinados; Logs de treinamento; Relatório comparativo de desempenho.

\subsection{Simulação e Validação}
\textbf{Duração:} 1 mês

\textbf{Atividades:}
\begin{itemize}[noitemsep]
    \item Execução de simulações de cenários emergenciais para testar o comportamento do modelo.
    \item Avaliação dos resultados utilizando as métricas definidas.
    \item Identificação de áreas críticas e padrões emergentes nas simulações.
    \item Ajustes preliminares nos modelos com base nos resultados das simulações.
\end{itemize}

\textbf{Entregas:} Resultados das simulações; Relatório de validação; Análise visual dos padrões identificados.

\subsection{Sistema de Alerta}
\textbf{Duração:} 1 mês

\textbf{Atividades:}
\begin{itemize}[noitemsep]
    \item Desenvolvimento do módulo de detecção de situações críticas.
    \item Implementação do sistema de geração de alertas baseado nas previsões do modelo.
    \item Integração do sistema de alerta com o framework de simulação.
    \item Testes de desempenho e usabilidade do módulo.
\end{itemize}

\textbf{Entregas:} Módulo de alerta implementado; Documentação do sistema; Relatório de testes.

\subsection{Avaliação e Otimização}
\textbf{Duração:} 1 mês

\textbf{Atividades:}
\begin{itemize}[noitemsep]
    \item Análise crítica dos resultados obtidos em todas as etapas anteriores.
    \item Identificação de limitações e oportunidades de melhoria.
    \item Otimização dos componentes críticos do framework.
    \item Validação final do sistema integrado.
\end{itemize}

\textbf{Entregas:} Relatório de avaliação; Framework otimizado; Documentação das melhorias implementadas.

\subsection{Redação e Revisão da Dissertação}
\textbf{Duração:} 3 meses

\textbf{Atividades:}
\begin{itemize}[noitemsep]
    \item Organização e síntese dos resultados obtidos.
    \item Redação dos capítulos da dissertação.
    \item Revisão do texto por pares e orientador.
    \item Elaboração da apresentação para defesa.
\end{itemize}

\textbf{Entregas:} Dissertação de Mestrado; Apresentação para defesa; Artigos científicos derivados do trabalho.

\section{Recursos Necessários}

Para a execução adequada do projeto, serão necessários os seguintes recursos:

\subsection{Recursos Computacionais}
\begin{itemize}[noitemsep]
    \item \textbf{Hardware:} Estação de trabalho com GPU compatível com CUDA (mínimo NVIDIA RTX 2080 ou equivalente) para treinamento dos modelos de \gls{GNN}; servidores com capacidade de processamento paralelo para simulações complexas.
    
    \item \textbf{Capacidade de Armazenamento:} Mínimo de 4TB para armazenamento de dados brutos, pré-processados e resultados de simulações.
    
    \item \textbf{Infraestrutura de Rede:} Conexão de alta velocidade para transferência eficiente de grandes volumes de dados.
\end{itemize}

\subsection{Recursos de Software}
\begin{itemize}[noitemsep]
    \item \textbf{Frameworks de Aprendizado Profundo:} PyTorch, PyTorch Geometric, TensorFlow.
    
    \item \textbf{Ferramentas de Geoprocessamento:} QGIS, ArcGIS, bibliotecas Python para manipulação de dados geoespaciais (GeoPandas, GDAL, Rasterio).
    
    \item \textbf{Bancos de Dados:} PostgreSQL com extensão PostGIS, MongoDB para dados não estruturados.
    
    \item \textbf{Ambientes de Desenvolvimento:} JupyterLab, VSCode, Git para controle de versão.
    
    \item \textbf{Ferramentas de Visualização:} Matplotlib, Plotly, Folium, NetworkX para visualização de grafos.
\end{itemize}

\subsection{Dados}
\begin{itemize}[noitemsep]
    \item \textbf{Registros de \gls{ERB}s:} Acesso a históricos de conexões de estações de rádio base (mediante parcerias com operadoras de telefonia).
    
    \item \textbf{Dados Geoespaciais:} Acesso a bases públicas (IBGE, OpenStreetMap, SRTM) e, quando necessário, aquisição de dados comerciais de alta resolução.
    
    \item \textbf{Informações Climatológicas:} Séries históricas do INMET e outras fontes meteorológicas.
    
    \item \textbf{Dados Socioeconômicos:} Acesso a microdados do Censo e pesquisas complementares do IBGE.
\end{itemize}

\subsection{Infraestrutura}
\begin{itemize}[noitemsep]
    \item \textbf{Laboratório:} Acesso a ambiente adequado para desenvolvimento e execução de simulações.
    
    \item \textbf{Recursos Bibliográficos:} Acesso a bases de dados científicas e periódicos especializados.
    
    \item \textbf{Suporte Técnico:} Apoio para configuração e manutenção dos ambientes computacionais.
\end{itemize}

\section{Considerações sobre o Cronograma}
O cronograma apresentado contempla as principais atividades do projeto, distribuídas ao longo de 15 meses. Foi elaborado prevendo-se eventuais ajustes conforme o desenvolvimento da pesquisa, com margem para acomodar desafios imprevistos que possam surgir durante o processo. 

A metodologia progressiva adotada garante que cada fase seja construída sobre os resultados da anterior, culminando no desenvolvimento de um framework robusto e validado para a modelagem preditiva de fluxos populacionais em cenários de emergência. Este planejamento também permite a geração de resultados parciais que poderão ser publicados ao longo do desenvolvimento do trabalho, contribuindo para a disseminação contínua do conhecimento gerado. 

% -----------------------------------------------------------------
% ELEMENTOS PÓS-TEXTUAIS
% -----------------------------------------------------------------
\postextual

% Glossário de Acrônimos
\printglossary[type=acronym,title=Lista de Acrônimos,toctitle=Lista de Acrônimos]

% Referências
\bibliographystyle{abntex2-alf}
\bibliography{bibliografia}

\end{document}